\documentclass{article}
\usepackage{lmodern}
\usepackage{ifluatex,ifdraft}
\usepackage[a6paper]{geometry}
\usepackage{parskip}
\usepackage{polyglossia}
\setmainlanguage[]{czech}
\usepackage{microtype}

\setlength{\emergencystretch}{3em}  % prevent overfull lines
\newcommand{\expl}[1]{{\sffamily #1}}

\RequirePackage{luatexbase,luacode}

\begin{document}
\section{Bad line breaks}

\expl{The follwing text has prepostions `O' and `k' at end of line which
  is incorrect according to Czech language typography standards:}

Mezi oblíbené dětské pohádky patří pohádky \emph{O Palečkovi, Alenka v říši
  divů} a \emph{Socialismem k lepším zítřkům.}

\expl{The follwing text is the same as above but parentheses are used to test
  that a preposition stay (or not stay, in this case) with following word
  even if the prepostion is preceeded with a parenthesis:}

Mezi oblíbené dětské pohádky patří pohádky \emph{[O Palečkovi], Alenka v říši
  divů} a \emph{Socialismem (k lepším zítřkům).}

Se správným nastavením fontů znak \% v `na 100 \%' skončí na další řádce.

\section{Better line breaks}

\expl{And here we add Lua code to penalize single letter words. It results in
following line layout:}

\ifluatex
      \begin{luacode}

      local prevent_single_letter = function (head)
        while head do
        -- glyph
        if head.id == 37 then
            if unicode.utf8.match(unicode.utf8.char(head.char),"[zZsSuUkKoOvVaAiI]") then
              -- only if we are at a one letter word
              if ((head.prev.id == 10
                   or (head.prev.id == 37
                       and unicode.utf8.match(unicode.utf8.char(head.prev.char),"[%[%]()%{%}]")))
                 and head.next.id == 10) then
                 local p = node.new("penalty")
                 p.penalty = 10000
                 node.insert_after(head,head,p)
               end
            end
          end
          head = head.next
        end
        return true
      end

      luatexbase.add_to_callback("pre_linebreak_filter",prevent_single_letter,"cssingleletter")

      local prevent_percents = function (head)
        while head do
        -- glyph
        if head.id == 37 then
            if unicode.utf8.match(unicode.utf8.char(head.char),"%%") then
              -- only if we are at percentage sign
              if (head.prev.id == 10
                  and head.prev.prev.id == 37
                  and unicode.utf8.match(unicode.utf8.char(head.prev.prev.char),"[0-9]")) then
                 local p = node.new("penalty")
                 p.penalty = 10000
                 node.insert_after(head.prev.prev,head.prev.prev,p)
               end
            end
          end
          head = head.next
        end
        return true
      end

      luatexbase.add_to_callback("pre_linebreak_filter",prevent_percents,"cspercents")

    \end{luacode}
\else
   \PackageError{The nosingleletter option only works with LuaTeX}
\fi

Mezi oblíbené dětské pohádky patří pohádky \emph{O Palečkovi, Alenka v říši
  divů} a \emph{Socialismem k lepším zítřkům.}

\expl{The follwing text is the same as above but parentheses are used to test
  that a preposition stay with following word even if the prepostion is
  preceeded with a parenthesis:}

Mezi oblíbené dětské pohádky patří pohádky \emph{[O Palečkovi], Alenka v říši
  divů} a \emph{Socialismem (k lepším zítřkům).}

Se správným nastavením fontů znak \% v `na 100 \%' skončí na další řádce.

\end{document}
